% Options for packages loaded elsewhere
\PassOptionsToPackage{unicode}{hyperref}
\PassOptionsToPackage{hyphens}{url}
%
\documentclass[
]{article}
\usepackage{amsmath,amssymb}
\usepackage{lmodern}
\usepackage{iftex}
\ifPDFTeX
  \usepackage[T1]{fontenc}
  \usepackage[utf8]{inputenc}
  \usepackage{textcomp} % provide euro and other symbols
\else % if luatex or xetex
  \usepackage{unicode-math}
  \defaultfontfeatures{Scale=MatchLowercase}
  \defaultfontfeatures[\rmfamily]{Ligatures=TeX,Scale=1}
\fi
% Use upquote if available, for straight quotes in verbatim environments
\IfFileExists{upquote.sty}{\usepackage{upquote}}{}
\IfFileExists{microtype.sty}{% use microtype if available
  \usepackage[]{microtype}
  \UseMicrotypeSet[protrusion]{basicmath} % disable protrusion for tt fonts
}{}
\makeatletter
\@ifundefined{KOMAClassName}{% if non-KOMA class
  \IfFileExists{parskip.sty}{%
    \usepackage{parskip}
  }{% else
    \setlength{\parindent}{0pt}
    \setlength{\parskip}{6pt plus 2pt minus 1pt}}
}{% if KOMA class
  \KOMAoptions{parskip=half}}
\makeatother
\usepackage{xcolor}
\IfFileExists{xurl.sty}{\usepackage{xurl}}{} % add URL line breaks if available
\IfFileExists{bookmark.sty}{\usepackage{bookmark}}{\usepackage{hyperref}}
\hypersetup{
  hidelinks,
  pdfcreator={LaTeX via pandoc}}
\urlstyle{same} % disable monospaced font for URLs
\usepackage[margin=1in]{geometry}
\usepackage{graphicx}
\makeatletter
\def\maxwidth{\ifdim\Gin@nat@width>\linewidth\linewidth\else\Gin@nat@width\fi}
\def\maxheight{\ifdim\Gin@nat@height>\textheight\textheight\else\Gin@nat@height\fi}
\makeatother
% Scale images if necessary, so that they will not overflow the page
% margins by default, and it is still possible to overwrite the defaults
% using explicit options in \includegraphics[width, height, ...]{}
\setkeys{Gin}{width=\maxwidth,height=\maxheight,keepaspectratio}
% Set default figure placement to htbp
\makeatletter
\def\fps@figure{htbp}
\makeatother
\setlength{\emergencystretch}{3em} % prevent overfull lines
\providecommand{\tightlist}{%
  \setlength{\itemsep}{0pt}\setlength{\parskip}{0pt}}
\setcounter{secnumdepth}{-\maxdimen} % remove section numbering
%%% Configuración del idioma
% \documentclass{book} %%  report
\usepackage{float}
\usepackage{longtable}
\usepackage[spanish]{babel}
\usepackage{titling}
\usepackage{eurosym}
% \floatsetup[table]{capposition=top}

\begin{titlepage}
  \begin{center}

    \begin{spacing}{0.51}
      % \textbf{\huge The Title of Image Comes Here}
      \textbf{\huge Regional report on complex forest characterization with NFI} 
      \vspace*{\fill}
    \end{spacing}
    
    \begin{spacing}{0.51}
      % \textbf
      {\large Spanish report}
      \vspace*{\fill}
    \end{spacing}

    \begin{figure}[ht]
      \centering      
      \includegraphics[scale=0.21]{./imagenes/COMFOR-logo.jpg}
    \end{figure}
    
    \begin{spacing}{0.51}
      \textcolor{other}
      iuFOR (Cristóbal Ordóñez, Irene Arrollo, Felipe Bravo  \\ [0.5cm]
      \vspace*{\fill}
    \end{spacing}
      
    \begin{spacing}{0.51}
      Marzo de 2018 \\
      \vspace*{\fill}
    \end{spacing}

  \end{center}

  \begin{figure}[]
    
    \centering
    \includegraphics[scale=0.273]{./imagenes/iuFOR.png}  
    \includegraphics[scale=1.5]{./imagenes/etsiiaa.png}
  \end{figure}

\end{titlepage}


% linestretch: 1.5                 # get some spacing between the lines you write
% bibliography: report.bib         # this is your bibliography file... it can be exported with Zotero, Medeley...
% biblio-style: apalike            # citations style
% link-citations: yes              # make your links clickable
\ifLuaTeX
  \usepackage{selnolig}  % disable illegal ligatures
\fi

\author{}
\date{\vspace{-2.5em}}

\begin{document}

\tableofcontents

\hypertarget{introducciuxf3n-y-marco-del-informe}{%
\section{Introducción y marco del
informe}\label{introducciuxf3n-y-marco-del-informe}}

En el presente informe se muestra el resultado de la valoración
económica de diversas alternativas selvícolas planteadas para masas de
\emph{Pinus pinaster} en el marco del proyecto del Grupo Operativo SIGCA
para madera de calidad de esta especie.

\hypertarget{objetivos}{%
\subsection{Objetivos}\label{objetivos}}

\begin{center}\includegraphics[width=300px]{./imagenes/UVa_logo_color_conLetras} \end{center}

El objetivo principal de este trabajo es describir la rentabilidad de
cuatro alternatívas selvícolas, con diferentes productos potenciales y
teniendo en cuenta la zona de plantación (interior o costa) y la calidad
de estación, bajo dos escenarios de precios del dinero y dos escenarios
de precios de los productos madereros.

\hypertarget{alternativas-selvuxedcolas}{%
\subsection{Alternativas selvícolas}\label{alternativas-selvuxedcolas}}

A continuación se describen las características de las cuatro
alternativas selvícolas teóricas que se han comparado.

\hypertarget{simulaciones-selvuxedcolas}{%
\subsection{Simulaciones selvícolas}\label{simulaciones-selvuxedcolas}}

Los escenarios descritos se han simulado según el modelo de crecimiento
elaborado por @Dieguez2009, y dentro del trabajo del Grupo Operativo
SIGCA. Los resultados de la simulación parten del año 0 de la planta y
tienen en cuenta la calidad de estación, tomada como la altura dominante
en metros a la edad de 20 años, y el área geográfica de Galicia en el
que se desarrolla, interior o costa.

A continuación se muestran los datos de las simulaciones, representados
por la evolución de volumen de madera producida, para cada uno de los
escenarios selvícolas, zona y calidad de estación.

\hypertarget{evoluciuxf3n-de-volumen-producido-de-madera-de-calidad.}{%
\subsubsection{Evolución de volumen producido de madera de
calidad.}\label{evoluciuxf3n-de-volumen-producido-de-madera-de-calidad.}}

Dado que el objetivo del grupo operativo esta centrado en la producción
de madera de calidad, se ha simulado de forma separada la que puede
destinarse a industrias que suponen un mayor valor añadido, considerando
como tal el volumen, en m\textsuperscript{3}/ha, de madera procedente de
trozas de más de 20 cm de diámetro en punta delgada.

El marco de la presente evaluación ha considerado los siguientes
factores: i) diferentes destinos de la madera según calidad y dimensión,
ii) diferentes alternativas de gestión, iii) diferentes escenarios de
precios de la madera considerando cinco destinos del producto, iv)
gastos fijos de plantación y clareos, y gastos variables en función de
las operaciones realizadas, y v) distintos precios del dinero.

\hypertarget{destinos-posibles-de-la-madera}{%
\subsubsection{Destinos posibles de la
madera}\label{destinos-posibles-de-la-madera}}

Para la clasificación de la madera según su uso potencial y por lo tanto
el precio que se puede conseguir por la madera, se ha utilizado la
adaptación realizada por AGRESTA S.Coop. de la la norma UNE EN
1927-2:2008. Basandose en criterios de dimensiones mínimas, se
distinguen 5 calidades de madera posibles:

\hypertarget{calidad-a}{%
\paragraph{Calidad A}\label{calidad-a}}

Trozas de calidad elevada cuyo destino habitual es chapa o carpintería
de alta calidad. Las dimensiones mínimas son 3 m de longitud y 45 cm de
diámetro.

\hypertarget{calidad-b}{%
\paragraph{Calidad B}\label{calidad-b}}

Trozas de buena calidad cuyo destino habitual es sierra de calidad alta,
carpintería de segunda calidad o tabla. Las dimensiones mínimas son 2,5
m de longitud y 40 cm de diámetro.

\hypertarget{calidad-c}{%
\paragraph{Calidad C}\label{calidad-c}}

Trozas rectas con nudos (no demasiados) y hasta 30 cm de diámetro cuyo
destino habitual es sierra de calidad media: vigas, viguetas, machones y
tablas. Las dimensiones mínimas son 2,5 m de longitud y 25 cm de
diámetro.

\hypertarget{calidad-d}{%
\paragraph{Calidad D}\label{calidad-d}}

Trozas curvadas y con nudos cuyo destino habitual es sierra de baja
calidad, encofrado y canter. Las dimensiones mínimas son 2 m de longitud
y 20 cm de diámetro (18 para canter).

\hypertarget{calidad-e}{%
\paragraph{Calidad E}\label{calidad-e}}

Trozas no aptas para sierra por defectos graves o diámetro insuficiente
cuyo destino habitual es combustible o trituración. No hay dimensiones
mínimas.

\hypertarget{precios-de-la-madera}{%
\subsubsection{Precios de la madera}\label{precios-de-la-madera}}

Para el precio de la madera se han considerado 2 escenarios diferentes,
dependientes de las calidades de los diferentes tipos de producto que
permitan hacer una evaluación de los diferentes itinerarios selvícolas
planteados en proyecto SIGCA:

\hypertarget{escenario-1}{%
\paragraph{Escenario 1}\label{escenario-1}}

Los precios de la madera de ``baja calidad'', similares a los actuales o
suben ligeramente, los precios de madera de calidad bajan a medida que
ascendemos de calidad

\hypertarget{escenario-2}{%
\paragraph{Escenario 2}\label{escenario-2}}

Los precios de la madera de ``baja calidad'' suben y los de madera de
calidad también suben.

Precios estimados medios de madera en pie que se han testado son:

En cargadero habría que sumar unos 12 /Mg. Hay que tener en cuenta que
cuanto más delgada es la madera que se aproveche más cara será la corta
y el desembosque, pero también hay que tener en cuenta que en las
primeras claras se cuenta con la ventaja de abrir calles lo que permiten
ahorrar costes. Por este motivo podemos suponer este coste como valor
medio para todas. Si tenemos en cuenta que las unidades con las que
vamos a trabajar, por la simulación realizada, son m\textsuperscript{3},
necesitamos estimar la densidad de madera verde. Vamos a suponer que
está en valores de 0,88 Mg/m\textsuperscript{3}, por lo que el precio en
cargadero debería aumentarse en 10,56 /m\textsuperscript{3}.

\hypertarget{gastos-de-gestiuxf3n}{%
\subsubsection{Gastos de gestión}\label{gastos-de-gestiuxf3n}}

En el presente estudio se han tenido en cuenta dos tipos de gastos, los
fijos que dependen de la superficie tratada y los variables que dependen
de la producción obtenida. Para estos dos tipos de gastos se suponen las
siguientes cuantías:

\begin{enumerate}
\def\labelenumi{\arabic{enumi}.}
\tightlist
\item
  Gastos fijos. Son así considerados los que dependen de la superficie y
  pueden ser:
\end{enumerate}

\begin{itemize}
\item
  Gasto de plantación, que supone un unos 2200 /ha plantada, y se asigna
  al año 1, ya que suponemos que la planta es de 1 savia. Habría que
  sumar entre 250 y 300 /ha si se utilizase planta mejorada
  genéticamente.
\item
  Gasto de clareos, que incluye el calleado, clareo manual y trituración
  de restos, y supone unos 1100 /ha tratada. La apertura de calles (roza
  mecanizada tractor de cadenas) en el año cuarto o quinto suponen 268
  /ha y la selección de pies en las calles (clareo o rareo) manual con
  motodesborzadora hasta ajustar denisdades a 1100 pies/ha supone 837
  /ha (inlcuye posterior triturado de restos en las calles, mecanizado
  mediante martillos)
\end{itemize}

\hypertarget{subvenciones}{%
\subsubsection{Subvenciones}\label{subvenciones}}

Para soportar la inversión inicial de establecimiento y tratamientos no
autofinanciados es habitual que se puedan solicitar ayudas a las
entidades autonómicas correspondientes. En el caso que nos atañe, por
ser donde son más habituales estas plantaciones, nos vamos a fijar en
las bases de ayudas de Galicia en 2019 (@xunta2019). Según los
tratamientos realizados se pueden solicitar ayudas por:

\begin{itemize}
\tightlist
\item
  plantación: coniferas 1100 pies/ha dificultades medias: 1527 /ha
\item
  poda: poda baja hasta 2,20 m en 800 pies/ha: 650 /Ha
\item
  Clareos: reducción de densidad 30\%, selección de pies menos
  desarrollados, sin apertura de calles ni saca, mediante motosierra.
  850 /ha
\item
  otras actuaciones posibles por las que se puede solicitar subvención:

  \begin{itemize}
  \tightlist
  \item
    desbroces por calles: 268 /ha
  \item
    perimetros de cortafuegos: 350 /ha
  \item
    fajas auxiliares de desbroce frente a incendios: 378 /ha
  \end{itemize}
\end{itemize}

En el estudio realizado se van a suponer dos ingresos posibles por
subvención:

\begin{enumerate}
\def\labelenumi{\arabic{enumi}.}
\item
  El que corresponde por plantación. Se supone un ingreso por subvención
  en el año 3, dos años después de la solicitud, y una cuantía de 1527
  /ha plantada que se aplica a todos los escenarios selvícolas salvo a
  MG2 con regeneración natural.
\item
  El que corresponde por clareos precomerciales. Este tratamiento solo
  se realiza en masas con regeneración natural, y que corresponde
  exclusivamente a MG2rn. En este escenario se incluye el ingreso de
  subvención por clara, que se realiza en el año 5 y tiene efecto dos
  años después de solicitarla, en el año 7, y por una cuantía de 850 /ha
  aclarada.
\end{enumerate}

\hypertarget{tasa-de-descuento-e-interuxe9s-aplicado.}{%
\subsubsection{Tasa de descuento e interés
aplicado.}\label{tasa-de-descuento-e-interuxe9s-aplicado.}}

Además se va a realizar una comparación del valor esperado del suelo
(@DiazBalteiro1998) para cada una de las alternativas.

Según el @BE\_sf, el precio del dinero o interés legal desde 2016 hasta
2020 se ha mantenido en un 3\%, en los últimos debido a la prórroga de
los Presupuestos Generales. En el análisis de una inversión se analizan
los flujos de ingresos y gastos futuros con sus valores en el momento
actual, para lo que hay que utilizar una tasa de descuento o coste del
capital. El valor que se utiliza para esta tasa es muy controvertido
(Ver @DiazBalteiro1998) y afecta de forma muy importante en el resultado
del análisis de una inversión.

y escenarios de precio del dinero

según el @BE\_sf, el precio del dinero o interés legal desde 2016 hasta
2020 se ha mantenido en un 3\%, en los últimos debido a la prórroga de
los Presupuestos Generales.

En el análisis económico se prevé que pueda existir una variación por lo
que se analizará para el caso de que suba o baje en un punto el precio
del dinero, mostrando la actualización de las rentas suponiendo que sea
un 2\% y un 4\%.

\hypertarget{anuxe1lisis-econuxf3mico}{%
\section{Análisis económico}\label{anuxe1lisis-econuxf3mico}}

Este análisis pretende comparar el rendimiento económico de las
distintas alternativas selvícolas planteadas, y para intentar
conseguirlo vamos a seguir dos enfoques; El primero comparará el valor
actual de ingresos y gastos generados en un único turno, presentado de
forma gráfica para facilitar las comparaciones. En segundo lugar se
realizará una valoración del monte, suponiendo que se van a realizar
plantaciones de manera indefinida, lo que permitira obtener una renta
que será independiente del turno y podremos comparar mediante una serie
de tablas.

Para todo el anális tendremos en cuenta el valor de mercado del vuelo,
prescindiendo del valor del suelo, y solo el relativo a la producción
primaria directa (Principalmente madera, y madera de pequeño tambaño o
leñas). Además se considerará un caso genérico, con valores medios para
todas las variables, y el valor económico estará siempre referido a la
hectárea. Para todo el anális tendremos en cuenta el valor de mercado
del vuelo, actualizado al momento de la plantación, prescindiendo del
valor del suelo, y solo el relativo a la producción primaria directa
(Principalmente madera, y madera de pequeño tambaño o leñas). Además se
considerará un caso genérico, con valores medios para todas las
variables, y el valor económico estará siempre referido a la hectárea.

\hypertarget{descripciuxf3n-cuantitativa-de-las-alternativas-selvuxedcolas}{%
\subsection{Descripción cuantitativa de las alternativas
selvícolas}\label{descripciuxf3n-cuantitativa-de-las-alternativas-selvuxedcolas}}

De forma complementaria se ha realizado una comparación del Valor
Esperado del Suelo (@DiazBalteiro1998) para cada una de las alternativas
selvícolas y condiciones de la estación. En esta valoración, indicada
para masas regulares no ordenadas, nos va a permitir comparar las
distintas opciones selvícolas independientemente del turno de las
mismas.

\hypertarget{alternativas-selvuxedcolas-1}{%
\subsection{Alternativas
selvícolas}\label{alternativas-selvuxedcolas-1}}

Se han considerado las cuatro alternativas de gestión descritas
anteriormente. Para cada una de ellas se ha supuesto que hay un
porcentage de madera que puede ir destinado a cada calidad en cada una
de las intervenciones previstas. Se supone que toda la madera que se
aproveche y tenga entre 7 y 20 cm en punta delgada será considerada de
calidad E.

Con las proporciones de cada calidad se puede calcular el precio medio
que tendrá la madera, suponiendo que su destino es el mejor de los
posibles. Podemos calcular, para cada escenario de precios, el precio de
la madera gruesa (VCC20) y el de la madera fina (VCC7).

\hypertarget{escenario-selvuxedcola-estuxe1ndar-habitual-m2}{%
\subsubsection{Escenario selvícola estándar habitual
(M2)}\label{escenario-selvuxedcola-estuxe1ndar-habitual-m2}}

En el escenario selvícola M2 tendremos aprovechamiento de madera en 3
claras y en la corta final. Se indica en el Cuadro 2.1 la proporción de
volumen de madera esperable para cada calidad y en el Cuadro 2.2 los
precios según cada escenario.

En el escenario selvícola M4 tendremos aprovechamiento de madera en 1
claras y en la corta final. En el Cuadro 2.3 se pueden ver las
proporciones de madera por destinos en porcentaje y en el Cuadro 2.4 el
precio medio que puede obtenerse por la madera aprovechada si va a la
mejor opción de las posibles por su tamaño.

\hypertarget{escenario-selvuxedcola-sin-gestiuxf3n-m8}{%
\subsubsection{Escenario selvícola sin gestión
(M8)}\label{escenario-selvuxedcola-sin-gestiuxf3n-m8}}

El escenario selvícola de no gestión a partir de la plantación inicial,
con una densidad de 1250 pies/ha, debe presuponer que hay mortalidad
natural, al menos en las calidades de estación más pobres. Este efecto
puede ser del orden del 10-20 \% en número de pies de alre

A continuación se muestra el volumen total acumulado que se aprovecha,
para cada uno de los escenarios selvícolas, zona y calidad de estación,
en m\textsuperscript{3}/ha de madera. Se presentan las gráficas de
volumen total de más de 7 y 20 cm de diámetro en punta delgada y el
volumen de madera delgada, entre 7 y 20 cm, de forma que se pueda tener
una idea clara de cual es el volumen aprovechado que se va a
contabilizar en el análisis económico posterior.

\hypertarget{volumen-total-de-muxe1s-de-7-cm-en-punta-delgada}{%
\subsubsection{Volumen total de más de 7 cm en punta
delgada}\label{volumen-total-de-muxe1s-de-7-cm-en-punta-delgada}}

Se muestra a continuación en la Figura @ref(fig:plotVACAR) la
distribución del volumen acumulado aprovechado de más de 7 cm en punta
delgada, comparando las calidades de estación para cada régimen de
clara:

En segundo lugar, se muestra la misma variable, volumen acumulado
aprovechado de más de 7 cm en punta delgada, pero comparando los
regimenes de clara para cada calidad de estación:

\hypertarget{volumen-total-de-muxe1s-de-20-cm-en-punta-delgada}{%
\subsubsection{Volumen total de más de 20 cm en punta
delgada}\label{volumen-total-de-muxe1s-de-20-cm-en-punta-delgada}}

Se muestra a continuación la distribución del volumen acumulado
aprovechado de más de 20 cm en punta delgada, comparando las calidades
de estación para cada régimen de clara:

Igual que para la variable anterior, se muestra el volumen acumulado
aprovechado de más de 20 cm en punta delgada, pero comparando los
regimenes de clara para cada calidad de estación:

\hypertarget{valor-esperado-del-suelo}{%
\subsection{Valor esperado del suelo}\label{valor-esperado-del-suelo}}

Para poder hacer comparables de forma directa los escenarios debemos
hacer una valoración que tiene en cuenta un escenario de infinitos
ciclos con las mismas condiciones ecologicas y económicas. Esta
valoración se conoce como Valor Esperado del Suelo y su cálculo está
detallado en @DiazBalteiro1998.

A continuación se muestra el resultado para cada calidad de estación y
zona.

La principal conclusión que podemos obtener del presente análisis
económico es que la calidad de estación mínima que asegura beneficio
económico es en estaciones medias (altura dominante de 16 m a 20 años)
para cualquier escenario selvícola de precios y de interés del dinero.
En estaciones de calidades más bajas solo es posible obtener
rentabilidad si es posible obtener una buena regeneración natural y con
la ayuda de subvenciones para tratamientos no autofinanciables.

La siguiente conclusión es que el escenario selvícola MG2 es el mejor en
todas las calidades y escenarios, salvo para calidad 7 m con un interés
del 2\%. Con este régimen selvícola se consigue una mayor rentabilidad
que en el escenario M2, el segundo mejor en todos los casos, pero con la
ventaja de necesitar 5 años menos para conseguirlo y además se consigue
con una actuación selvícola menos.

Si comparamos los escenarios MG2 con y sin plantación, muestra que la
regeneración natural es más rentable, principalmente por el ahorro que
supone no tener que hacer la plantación, que es más cara que el clareo
que hay que hacer con regeneración natural. Esto supone que, sobre todo
para zonas de calidades altas, la plantación podría ser una mejor opción
si se utiliza planta mejorada genéticamente. Lamentablemente no existen
modelos de crecimiento que tengan en cuenta esta opción, pero el escaso
margen de diferencia que hay en este análisis hace suponer que el
incremento de precio de esa planta mejorada puede traducirse en una
mejora de la rentabilidad, si bien no podemos cuantificar esa mejora.

En zonas de alta calidad de estación parece que las previsiones de
volumen aprovechable son muy buenas, quizá excesivas, sobre todo para el
escenario M8 sin ninguna intervención selvícola. Esto puede hacer pensar
que es necesaria una simulación selvícola más precisa, sobre todo en
este escenario sin selvicultura en el que incluir el efecto de la
mortalidad natural parece obligado.

En todo caso, es necesario tener en cuenta que para este análisis se han
tenido en cuenta valores medios en cuanto a la distribución y producción
del vuelo, y costes medios de los aprovechamientos, sin tener en cuenta
la proximidad a la industria destino ni la dificultad de acceso al monte
o la facilidad de mecanización del propio aprovechamiento. Estas
consideraciones pueden hacer que un sitio de baja calidad de estación
pueda convertirse en una buena opción para hacer una plantación de
\emph{Pinus pinaster} que no lo sería en una ubicación o con unas
condiciones de pendiente o accesibilidad muy adversas.

Por el mismo motivo también es importante tener en cuenta que un área
con buena calidad de estación pero condiciones de accesibilidad o
topográficas adversas, pueden hacer que una plantación que tiene una
previsión de rentabilidad excelente, se convierta en una explotación que
finalmente vea disminuidas las rentas a final de turno.

Siempre deberemos tener en cuenta que el análisis de rentabilidad
realizado se ha hecho con con datos medios, tanto los relativos a la
productividad como los costes y precios del producto final, por lo que
los resultados deben tratarse con precaución; se ofrecen conclusiones
que sirve para un caso genérico, y se pueden tener en cuenta como una
primera aproximación, pero en inversiones con excaso margen de beneficio
como son las explotaciones forestales, es necesario estudiar cada monte
concreto para asegurar en la medida de lo posible la inversión que se
realize.

\hypertarget{referencias}{%
\section{Referencias}\label{referencias}}

\end{document}
